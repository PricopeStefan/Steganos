
\chapter{Image file formats and steganography techniques}

\section{Introduction}

\setlength\columnsep{20pt}
\begin{multicols*}{2}
An image is a two-dimensional representation depicting any possible subject conceivable by human imagination, captured using an optical device (such as a camera or a telescope) or a natural object (human eyes). The image can then be rendered and displayed for other people to see either manually (by painting, carving etc.) or automatically (by using a computer). In this chapter we will focus on images captured using digital optical devices that are rendered automatically. The correct term for them is digital images, but throughout the rest of the paper they will be reffered to as images for convenience.

\begin{figure}[H]
    \centering
    \includegraphics[width=5cm,height=5cm,keepaspectratio]{pics/lenna}
    \caption{Lenna - Classic example of a digital image}
    \label{Lenna}
\end{figure}

Computers are programmed to do operations in a clear sequential way and this rule doesn't change when working with pictures. In order for a computer to be able to render an image, it needs to know some general metadata information about the photo, such as the width and height, as well as the data bytes of the image. These bytes are the actual representation of the picture which compose the two-dimensional pixel map\footnote{This is true for a lossless format, where each pixel is stored in memory. It is not exactly the case for lossy formats such as JPEG where the image goes through processing before being rendered. More information later in the chapter.}. A pixel is the smallest unit that a computer monitor can read and display. The pixel color is the result of merging the different color channels which compose the picture (such as RGB, YUV, YCbCr etc.). Here is an example of the entire process - let's assume that from the image data bytes the first 3 bytes have the decimal values 20, 127, 250 and that it uses the RGB color model. This means that when the computer will have to render the image, the first pixel will have the red component equal to 20 (0x14), the green equal to 127 (0x7F), and the blue equal to 250 (0xFA), in what will finally be interpreted as \#147FFA by the monitor (variation of light blue). 

\begin{figure}[H]
    \centering
    \includegraphics[width=8cm,height=2.15cm,keepaspectratio]{pics/how_a_pixel_works}
    \caption{How 3 colors channels build the pixel}
    \label{Pixel Creation}
\end{figure}

It is important to note that most of these developments have been done in a time where the maximum storage was extremely limited and not very fast, very different from what it is today. In order to save some space they looked into different compression algorithms to apply to the byte streams and today they fall into two distinict categories: the lossless algorithms are the ones that compress all the original information without destroying any of it, while on the other hand there are the lossy algorithms which are able to identify which information is useless and delete it accordingly. This concept also applies to file formats and we will see in later chapters more concrete examples. 

With all of this information in mind, we can now procede to discussing the most commonly .

\section{BitMap Picture (BMP)} \label{BMP_Explained_Chapter}

The BMP file format, also known as the device independent bitmap file format(DIB), bitmap image file or just bitmap, is a lossless\footnote{It is true that the format specification standards supports compression but further research reveals that to this day it only supports lossless types of compression, such as Huffman or Run Length Encoding compression algorithms.} image file format originally designed by Microsoft back in 1986 in order to store two-dimensional digital images on their Windows operating system. Over the years it has developed plenty of variations and extensions that were based on the original specification but this paper chooses to focus only on the most common available ones.





\subsection{Image sub-block scrambling using the BMP format}

\section{Portable Network Graphics (PNG)} \label{PNG_Explained_Chapter}

\section{Joint Photographic Experts Group (JPEG)} \label{JPEG_Explained_Chapter}

\end{multicols*}