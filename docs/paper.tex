\documentclass[notitlepage]{report}
\usepackage{graphicx, multicol, float}
\usepackage[margin=1in]{geometry}
\usepackage{indentfirst}
\usepackage{titling}
\usepackage[dvipsnames]{xcolor}
\usepackage{amsmath}
\usepackage{amssymb}
\usepackage[normalem]{ulem}
\usepackage{listings}
\usepackage{subcaption}
\usepackage{xcolor}
\usepackage{url}
\useunder{\uline}{\ul}{}
\begin{document}


\begin{center}
        \vspace*{1cm}
        
        \LARGE
        \textbf{UNIVERSITATEA BABEȘ-BOLYAI CLUJ-NAPOCA \\ FACULTATEA DE MATEMATICĂ ȘI INFORMATICĂ \\ SPECIALIZAREA: INFORMATICĂ ROMÂNĂ}
        \vspace{1cm}\\

        \vfill
        \LARGE
        \textbf{LUCRARE DE LICENȚĂ} \\
	  \Large
	  \textbf{Steganografie în mediul digital}
        \vspace{1cm}

        
        \vfill
        
        \textbf{Absolvent:\\ Pricope Ștefan-Cristian}
        \vspace{0.1cm}

        \textbf{Profesor îndrumător:\\ Dr. Suciu Mihai, Conferențiar Universitar}\\
        \vspace{1cm}        

	  \Large
	  \textbf{CLUJ-NAPOCA, 2020}
        \vspace{0.3cm}
\end{center}


\clearpage
%Abstract - small description of the paper
\begin{abstract}
Steganography is the science of concealing a piece of information within another piece of information without affecting the latter in a noticeable way and therefore alerting any intruders of the existence of the former. 
This thesis presents both existing and new ways of embedding computer files and data into different digital multimedia formats as covert as possible while still allowing the encoded information to be retrieved at a later time without any significant losses. It also       presents the structure of some of the most common multimedia files that are used in the modern day digital world and are viable candidates for the role of the cover file in the steganographic process.

This work is the result of my own activity. I have neither given nor received unauthorized assistance on this work.
\end{abstract}

\begin{figure}[hb!]
  \includegraphics[width=3cm]{pics/semnatura}
\end{figure}

%Aici vine cuprinsu
\tableofcontents{}

%Aici incep capitolele
%Abstract - small description of the paper
\begin{abstract}
Steganography is the combination between the 2 Greek words : steganós (which means concealed) and graphia(which means writing) \cite{steganography-origin} and has been a method that humanity has used to hide information for more than 2 millenniums \cite{steganography-history}.
This paper presents both existing and new ways of embedding computer files and data into different digital multimedia formats as covert as possible and allowing the encoded information to be retrieved at a later time without any losses.
\end{abstract}

\section[Introduction to computer Steganography]{Introduction}

%\begin{figure}[H]
%    \centering
%    \includegraphics[]{pics/bmp_scan_line}
%    \caption{How BMP works}
%    \label{simulationfigure}
%\end{figure}

\begin{multicols}{2}

%introduction section
\subsection{Subsection Heading Here}
%subsection text
\end{multicols}


\begin{thebibliography}{9}
\bibitem{steganography-origin}
Merriam-Webster dictionary.
\\\texttt{https://www.merriam-webster.com/dictionary/steganography}

\bibitem{steganography-history}
Petitcolas FAP, Anderson RJ and Kuhn MG.
\textit{Information Hiding - A Survey}.
Proceedings of the IEEE, special issue on protection of multimedia content, 1999.
% http://www.creangel.com/papers/steganografia.pdf
\bibitem{seeing-the-unseen} 
Johnson Neil and Jajodia Susil.
\textit{Exploring Steganography : Seeing the Unseen}. 
Los Alamitos, IEEE Computer Society, 1998.

%http://www.academia.edu/download/54323461/EN_-_Image_Steganography_Overview.pdf
\bibitem{overview-image-steganography} 
T. Morkel, J.H.P. Eloff and M.S. Olivier. 
\textit{An overview of image steganography}.
University of Pretoria, ICSA Research Group, 2005.

%http://honeyman.org/u/provos/papers/practical.pdf
\bibitem{hide-and-seek} 
Niels Provos and Peter Honeyman.
\textit{Hide and Seek: An Introduction to Steganography}.
University of Michigan, IEEE Computer Society, 2003

\end{thebibliography}



\chapter{Steganography methods}


\section{Least Significant Bit(LSB)}
\begin{multicols*}{2}
\subsection{Sequential}
\setlength\columnsep{20pt}

Least Significant Bit or LSB is by far the most used method when talking about any type of steganography. Given that the smallest unit a computer can understand and process is usually a byte, altering only the least significant bit will not change the transmitted information in a noticeable way to any external parties. It is much easier to showcase what a byte contains and what the LSB change implies and how it works. A byte contains 8 bits, so this means that the values a byte can take range anywhere from 0 to 255 (inclusive)\footnote{This is the case for unsigned bytes, but given that we are talking about a method that only deals with the least significant bit, we can safely ignore the most significant bit, also known as the sign bit.}. Let's assume that we have an array of 4 random values in consecutive memory : 217, 127, 100, 62 (all values are in decimal), each stored on exactly one byte, and that we want to hide our grade in Numerical Analysis from our parents (in this case a 3) using a LSB substitution. The process would be something like this :

\begin{figure}[H]
    \centering
    \includegraphics[width=2.8cm,keepaspectratio]{pics/how_lsb_works}
    \caption{How the sequential Least Significant Bit change works}
    \label{LSB}
\end{figure}

As we can see from Figure 2.1, we have succesfully altered the least significant bit of the first 3 bytes of the stream in order to hide our grade : 217 became 216 when we changed the last bit from 1 to 0, 127 was unchanged because it already had the last bit set, and the third byte became 101 after toggling the final bit. Furthermore, the rest of the stream (the fourth byte, 62) was not affected because we already hid the entirety of our secret message. While this is great because we only hide exactly as much as we need and not a byte more, we have a high risk of corrupting the hidden message in case our cover image gets compressed or loses even a single byte when sent over a network. Basically, we are trading data redundancy in order to get simplicity and efficiency.

Sequential LSB insertion is the simplest and most common way of embedding any kind of information into a byte stream that is then shared. It has been thoroughly discussed by a great deal of researchers and has been the subject of many papers where it was analyzed and benchmarked\cite{seeing-the-unseen}\cite{hide-and-seek} . Being the most popular technique also means that any flaws the method has are widely documented and showcased. Steganalysis\footnote{Steganalysis is the study of steganographic methods, including but not limited to : differences in file sizes or in color histograms, secret message redundancy, embedding capacity and performance etc.} performed on outputs created using this algorithm has shown that it is unreliable to stay undetected if an outside party intercepted the message containing the cover file\cite{attack-on-steganography}. Furthermore, doing a simple reverse engineering on the algorithm reveals even more issues with this naive encoding : if a single bit that is part of the secret message was flipped from the cover file byte stream, the message would become corrupted and the original secret would be lost forever. This means that sequential LSB insertion is not resistant at all to any form of lossy compression where some of the original data may be lost in order to reduce the used disk space because it would lose most, if not all, of the embedded file information. 

Furthermore, it is extremely easy to compute the carrier storage, i.e. how many bytes we are able to hide into the cover file or in other words, the maximum size of the secret message that we can succesfully embed without losing anything while still keeping a covert profile. Assuming $CDS$ or Cover Data Size (how many bytes are actually used to store the pixel information, no metadata information or chunk headers or anything like that), then the $MMS$ or Maximum Message Size would be equal to

\[ MMS = [CDS \ / \ 8] \ bytes \]

or the integer part of CDS divided by 8. This should come as no surprise since sequential LSB is altering $1/8$ (an eighth) of each byte when embedding a bit of the secret information so the maximum capacity makes sense to also be $1/8$.


\subsection{Scrambling}
As documented earlier, sequential LSB insertion algorithm has a decent number of flaws so it was needed to develop some new techniques that are not relying as much on the cover file not losing a few bytes or undergoing a compression algorithm. In other words, it needed to introduce a few redundancies to ensure that the secret message wouldn't be lost as easily and that the message was not written in a sequential and direct order. They achieved this by not just changing the least significant bit in a sequential order, but by writing in an apparently random order (in simpler terms, scrambled) that could be reproduced by having the right key or by using the same algorithm in order to retrieve the embedded information. In this subsection we will discuss a few of the most common scrambling techniques and introduce a new one as well.

The most popular methods used for scrambling secret messages into various covers usually choose to simply ignore the entire data stream and only focus on a specific subsection and choose that as the carrier environment. After a smaller subsection is chosen (it can still be the entire actual data part of the cover, it's not an actual rule), we will have to generate the order in which we will write the message information. As mentioned before, this is derived using a key known only to the sender and the receiver that can be shared between the 2 parties using another transmission environment, preferably one that is encrypted and safe. The main logic is to use that passkey as a seed in a valid pseudo random engine when generating the order so that there are no collisions and that only the right key will produce the right order. 

There is also an option for when there is no safe method of transmitting a key and that is to scramble the secret message within a certain order that is not sequential. But as long as the receiver and transmitter have no way of communicating the algorithm used in embedding the message (can also assume this because they have no way of sharing the key), this option is useless but is still interesting to look into because they are a variation of the other mentioned option. Having no key to generate the order makes this option easier to showcase(since we don't have to also simulate a pseudo-random engine and a seed). Let's assume we have a byte stream of random data\footnote{Now the actual data meaning can be safely ignored because we only work on the LSB and we have seen in the previous chapter it does not alter the data in a significant way such that an intruder might notice something is wrong.} and we want to again hide a grade from our parents, in this case a 6 (or 110 in binary). Instead of hiding the grade sequentially, the bits will be hidden into the last bit of every third byte and it will end up looking something like this :

\begin{figure}[H]
    \centering
    \includegraphics[width=5cm,keepaspectratio]{pics/scrambling_example}
    \caption{An example of scrambling LSB insertion}
    \label{Scrambled LSB}
\end{figure}

The key difference from sequential LSB insertion is that as long as the sender and receiver are aware of the used algorithm, any external parties will not be aware of the secret embedded message. Furthermore, this method has proven very useful because there are infinite ways of scrambling a message into the cover file without alerting any possible intruders and it ends up being an extremely hard guessing game for the attackers in their goal to extract the information.

The carrier capacity appears initially to be the same, but it is very important to remember that the scrambling algorithm used will usually work on a subset of the cover data bytes, not the entirety of it (just like in figure 2.2 we used only each third byte to store the information). Using the same notations as in the previous chapter we get that 

\[ MMS \leq [CDS \ / \ 8] \ bytes \]

so in most cases, the scrambled MMS will be smaller than the sequential MMS. However it is very important to note that this decrease in size comes with a great increase in data security and message safety.

This paper also introduces a new type of scrambling algorithm created by the authors that only works on lossless image formats that do not use any kind of interlacing when rendering the picture. It relies on scrambling the secret message into the image sub-blocks in a specific order that can only be deduced by having the right passkey. More information on this method is presented in chapter 3.2.1 after the introduction of image basics.


\section{Metadata encoding}
The word metadata was formed from combining the word "data" with the prefix "meta-" and is used to describe a special type of data that has information about other types of data\cite{metadata-origin}. In simpler terms, it means "data about data" and it keeps the same meaning in the digital world. It is much easier to visualize and understand the concept of metadata using a simple example : let's take a picture stored on a computer.

\begin{figure}[H]
    \centering
    \includegraphics[width=2.3cm,keepaspectratio]{pics/goose_file}
    \caption{A simple image file}
    \label{Goose Image File}
\end{figure}

As noted in the introductory chapter, every file can be seen as a byte stream. However it is very important to keep in mind that those bytes don't represent only the image data, the pixels seen on the screen, they are much more. They also contain information about the camera used to take the photo, the location where it was taken, when the file was created, when it was modified, etc. All of the aforementioned information forms the metadata. It varies from file format to file format where they store this information in the byte stream, how many bytes are allocated for each piece of information or if it even has any effects on the actual file data. Most of the time metadata fields are only parsed by the renderer software and are mostly hidden to the user, but there are a few ways of viewing the information: 
\begin{itemize}
  \item using a hex editor to view the raw bytes of the file and then mapping those bytes to the publicly available file format specifier - an international approved paper which specifies the meaning of the bytes in the file binary stream
  \item using the operating system to view more properties about the file, not just the actual data interpreted and displayed to the end user
  \item using a third party tool which already knows the mapping and meaning of each sequence of bytes in the file binary stream and can succesfully parse the metadata
\end{itemize}

\begin{figure}[H]
    \centering
    \includegraphics[width=8cm,keepaspectratio]{pics/exiftool_file_metadata}
    \caption{Example of using a tool to read file metadata}
    \label{Goose Image File Metadata using Exiftool}
\end{figure}

The focus of this sub-chapter will be on the metadata fields that don't necessarily have any important effect on the data representation and theoretically could be altered, such as any comments from the author or any contact information. In the case of an image, changing important metadata fields such as the width or height of the picture are not very covert methods of embedding any information at all, proving that not all fields are equally important. We are left with the more \textit{useless} metadata fields, but the good news is that most file formats allow these fields to have a variable size which is perfect for any steganographic purposes because it removes the size constraint of the secret message. While in theory this allows for $\infty$ MMS, there are a few limitations in place that significantly reduce that number: 
\begin{itemize}
  \item computers have a limited amount of storage space, in most modern day computers that would be about one terabyte. This means that the MMS will certainly be smaller than that.
  \item most file formats that allow metadata fields to be embedded in the byte stream of the file by an external party also have a sequence of bytes that indicates how long that metadata field is (how many bytes it contains). In most cases that sequence is 4 bytes long so this usually results in a MMS of $2^{4*8} = 2^{32} \approx 4.29\ gigabytes$.
  \item in the pursuit of keeping the existence of a secret inside the cover file as covert as possible, the MMS is again limited by the CDS. This happens because the size of the cover file is almost always displayed to any parties without any interactions and it needs to not raise any suspicions or attract unwanted attention to the actual contents of the file. To give an example, it would be very weird to see a simple image in FULL HD resolution have a size of two gigabytes and will almost certainly alert any intruders. It is extremely hard to say an exact MMS based on this limitation because it depends on the original cover file size and it should be relative to that number. It is recommended that $MMS \leq  50\% * CDS$.
\end{itemize}

In the end, metadata secret encoding is the most trivial method of embedding secret information into different kinds of cover files. However, this comes with a great cost: almost every single tool that specializes in extracting metadata will be able to detect and identify our message without too much hassle because the final cover file still has to respect the format specification.

\begin{figure}[H]
    \centering
    \includegraphics[width=8cm,keepaspectratio]{pics/metadata_message_identified}
    \caption{Example of message hidden in a file in the metadata section}
    \label{MetadataMessageExample}
\end{figure}

\section{Unused space embedding}
sdfasfasd

\end{multicols*}

\chapter{Image file formats and steganography techniques}

\section{Introduction}

\setlength\columnsep{20pt}
\begin{multicols}{2}
An image is a two-dimensional representation depicting any possible subject conceivable by human imagination, captured using an optical device (such as a camera or a telescope) or a natural object (human eyes). The image can then be rendered and displayed for other people to see either manually (by painting, carving etc.) or automatically (by using a computer). In this chapter we will focus on images captured using digital optical devices that are rendered automatically. The correct term for them is digital images, but throughout the rest of the paper they will be reffered to as images for convenience.

\begin{figure}[H]
    \centering
    \includegraphics[width=5cm,height=5cm,keepaspectratio]{pics/lenna}
    \caption{Lenna - Classic example of a digital image}
    \label{Lenna}
\end{figure}

Computers are programmed to do operations in a clear sequential way and this rule doesn't change when working with pictures. In order for a computer to be able to render an image, it needs to know some general metadata information about the photo, such as the width and height, as well as the data bytes of the image. These bytes are the actual representation of the picture which compose the two-dimensional pixel map\footnote{This is true for a lossless format, where each pixel is stored in memory. It is not exactly the case for lossy formats such as JPEG where the image goes through processing before being rendered. More information later in the chapter.}. A pixel is the smallest unit that a computer monitor can read and display. The pixel color is the result of merging the different color channels which compose the picture (such as RGB, YUV, YCbCr etc.). Here is an example of the entire process - let's assume that from the image data bytes the first 3 bytes have the decimal values 20, 127, 250 and that it uses the RGB color model. This means that when the computer will have to render the image, the first pixel will have the red component equal to 20 (0x14), the green equal to 127 (0x7F), and the blue equal to 250 (0xFA), in what will finally be interpreted as \#147FFA by the monitor (variation of light blue). 

\begin{figure}[H]
    \centering
    \includegraphics[width=8cm,height=2.15cm,keepaspectratio]{pics/how_a_pixel_works}
    \caption{How 3 colors channels build the pixel}
    \label{Pixel Creation}
\end{figure}

It is important to note that most of these developments have been done in a time where the maximum storage was extremely limited and not very fast, very different from what it is today. In order to save some space they looked into different compression algorithms to apply to the byte streams and today they fall into two distinict categories: the lossless algorithms are the ones that compress all the original information without destroying any of it, while on the other hand there are the lossy algorithms which are able to identify which information is useless and delete it accordingly. This concept also applies to file formats and we will see in later chapters more concrete examples. 

With all of this information in mind, we can now procede to discussing the most commonly formats commonly used in today's.

\section{BitMap Picture (BMP)} \label{BMP_Explained_Chapter}

The BMP file format, also known as the device independent bitmap file format(DIB), bitmap image file or just bitmap, is a lossless\footnote{It is true that the format specification standards supports compression but further research reveals that currently it only supports lossless types of compression, such as the Huffman or Run Length Encoding algorithms.} image file format originally designed by Microsoft back in 1986 in order to store two-dimensional digital images on their Windows operating system. Over the years it has developed plenty of variations and extensions that were based on the original specification but this paper will focus only on the most common available ones, no extended versions that are looking to improve the format since they do not add anything interesting or new to the way the format stores the data thus affecting the steganography algorithms.

As with almost every file format, the final BMP byte stream can be seen as a result of the merge between the BMP header which contains metadata about the file and the BMP data which is the actual pixel information. As we can see from table \ref{BMP_Header_Table}, the BMP header stores a lot of important information about the image that is useful for any rendering software while making sure to allocate enough memory to be able to display the picture on the screen and other essential steps. It is also important to note that all the structures seen in the BMP header use the little-endian format for representation and are usually more troublesome on the systems that have the default set as big-endian.
\end{multicols}

 \begin{center}
	\begin{tabular}{|l|l|l|l|}
		\hline
		\textbf{Information} & \textbf{Size} & \textbf{Offset} & \textbf{Description} \\ \hline
		Signature & 2 bytes & 0x00 & Two chars, 'B' and 'M' \\ \hline
		File size & 4 bytes & 0x02 & Total file size in bytes \\ \hline
		Reserved & 4 bytes & 0x06 & Unused space \\ \hline
		Data offset & 4 bytes & 0x0A & Offset to get to the actual BMP data \\ \hline
		Size & 4 bytes & 0x0E & Size of the left header information \\ \hline
		Width & 4 bytes & 0x12 & Horizontal size of the image \\ \hline
		Height & 4 bytes & 0x16 & Vertical size of the image \\ \hline
		Planes & 2 bytes & 0x1A & Amount of image planes \\ \hline
		Bits Per Pixel & 2 bytes & 0x1C & How many bits are used to represent each pixel \\ \hline
		Compression & 4 bytes & 0x1E & Indicates the type of compression used \\ \hline
		Image size & 4 bytes & 0x22 & The size of the compressed image, can be 0 \\ \hline
		X pixels per Meter & 4 bytes & 0x26 & Horizontal resolution in pixels/meter \\ \hline
		Y pixels Per Meter & 4 bytes & 0x2A & Vertical resolution in pixels/meter \\ \hline
		Colors Used & 4 bytes & 0x2E & \begin{tabular}[c]{@{}l@{}}Number of actually used colors\\ (based on Bits Per Pixel)\end{tabular} \\ \hline
		Important Colors & 4 bytes & 0x32 & Number of important colors (usually all) \\ \hline
	\end{tabular}
	\label{BMP_Header_Table}
 \end{center}

\begin{multicols*}{2}
Analyzing the obligatory BMP header fields we realise that most of them are useless for any steganographic purposes mainly because they can't be altered without having major consequences on the renderer software but there are still a few interesting ones left:
\begin{itemize}
  \item The 4 bytes that are reserved and unused could be very well put to use by using the methods presented in chapter \ref{Unused_Space_Chapter}, so we can use either this space to send parts of a message over multiple BMP files, or we can store the secret message size in these 4 bytes and write the secret after the actual image data has ended.
  \item Data offset could be useful because some renderers use this field to indicate the offset of the actual image data and just skip any other irelevant metadata information, allowing us to hide information in those fields.
  \item The width and the height of the image usually are altered to hide bottom parts of the image that may contain hidden information or by masking the result of the merge of two images and just showing the top one. Let's take for example this image of the sun that is obviously missing a part for demonstration purposes.

\begin{figure}[H]
    \centering
    \includegraphics[width=4cm,keepaspectratio]{pics/height_modification_steganography_cut}
    \caption{Sun that is clearly missing something}
    \label{Sun_Missing_Part}
\end{figure}

However were we to increase by trial and error the height of the image to see if there is any more bytes to render, we would notice that there really is a message hidden with those bytes that some steganalysis softwares would detect but any renderer software such as the Windows Media Viewer would always miss.

\begin{figure}[H]
    \centering
    \includegraphics[width=4cm,keepaspectratio]{pics/height_modification_steganography_original}
    \caption{Complete sun with the hidden message}
    \label{Sun_Original}
\end{figure}
\end{itemize}

\subsection{Image sub-block scrambling using the BMP format}

\section{Portable Network Graphics (PNG)} \label{PNG_Explained_Chapter}

\section{Joint Photographic Experts Group (JPEG)} \label{JPEG_Explained_Chapter}

\end{multicols*}

\chapter{Audio file formats and steganography techniques}

\begin{multicols*}{2}
\section{Introduction}
Sound is the result of a vibration created by a phenomenon that propagates through a transmission environment, ending up getting interpreted by our brain. However since this process happens in the physical world it is entirely analog so it would be impossible to store it on modern day devices which can only understand digital formats. Luckily the fast evolution of computers also brought conversion techniques in order to switch between analog sounds and digital sounds seamlessly, without any noticeable loss to the human ear. Using these methods we have gained the ability to store audio files in a digital format so it was only logical that several different file formats will be created to fit our needs. In this chapter we will discuss in greater detail about how the analog data is actually stored in the digital format and what are the most common extensions used for storing digital audio files.

We mentioned earlier that it is impossible to store analog signals in a digital environment. The solution to this problem is to convert the audio signal which can be represented as a continuous-time function into a discrete-time function using a process called sampling. 

\begin{figure}[H]
    \centering
    \includegraphics[width=7.7cm,keepaspectratio]{pics/Sampling-of-audio-signal.png}
    \caption{Converting a continuous signal into a discrete one\cite{real_time_audio_steganography}.}
    \label{sampling-graphic-example}
\end{figure}

The sampling process can be observed in figure \ref{sampling-graphic-example}. We can now introduce some new terms that we will use throughout the rest of the chapter:
\begin{itemize}
	\item \textbf{Sampling rate} is the number of samples taken per second, or in other terms, how many discrete values we store for each second of the audio signal. The measurement unit for sampling rate is Hertz (Hz for short) and some of the most common values are 44100, 48000 and some of their multiples.
	\item \textbf{Bit depth} is the number of bits used to store a single sample after having it converted to a discrete value. The most common bit depths are 8, 16, and 24 which allow for 256, 65536, and 16777216 different values. 
	\item \textbf{Audio channel} is the term used to describe the sequence of bytes that represent sampled audio signals. An audio file can have multiple channels to better simulate the sound accuracy and origin in a limited environment. Files with one audio channel are called mono, with two they become stereo and any more channels makes them surround. However, no matter the number of channels, usually all of them are equal in length and the samples from each channel are played simultaneously.
\end{itemize}

In the steganography field, the most common configuration that accepts alterations to the sampled data without losing any noticeable quality is a sampling rate of 44.1kHz with a bit depth of 16 and any number of audio channels, so this the ideal format that we will use throughout the rest of the chapter. The reason why this configuration is favored so much is because of the popularity that came with the invention of CDs and MP3s which used it as a default. Furthermore, any changes made to the sampled data are usually small enough that they will not be noticeable according to the Nyquist-Shannon sampling theorem \cite{Shannon1949}, which is used to compute the condition such that the conversion from a continuous signal to a discrete one will capture all the relevant information. Using the aforementioned theorem and armed with the knowledge that the human physiology enables us to hear audio signals ranging from 20Hz to 20kHz, we can see why the 44.1kHz sampling rate is ideal in audio steganography.

\section{Additional techniques used in audio steganography}
\subsection{Frequency domain steganography}
So far in this thesis we have talked about what can only be classified as spatial domain steganography, like how a pixel of an image is composed of bytes and that altering those byte values in a smart way allows for message embedding or how we can use the file specifications to our advantage and hide information in the file metadata or after the offset where all renderer programs will stop parsing. All of the previous examples deal ony with the spatial domain of the format because they work on the raw bytes and consider each of them to be completely individual and self-sustaining entities that can be altered for steganographic purposes. However, there is another domain that works differently than the spatial one and it is called the frequency domain. In this domain the final representation of the stored data is done by combining the entire range of frequencies into the equivalent signal, usually by using a transform function, the most common being the Fourier transform. The advantage of the frequency domain is that it helps split a signal of any type into clear and distinct sinusoids that are easier to perform complex tasks on. 

In figure \ref{time_vs_frequency_comparison} we can see the differences between the time domain and the frequency domain: the former evolves over time and is highly irregular in most cases(in real world there will never be a perfect sinusoidal signal and will be more similar to the last example), while the latter is easier to manipulate and identify, and through the aforementioned transform functions can be converted back into the time domain.
\begin{figure}[H]
    \centering
    \includegraphics[width=7.7cm,keepaspectratio]{pics/audio_chapter/time_vs_frequency_domains.png}
    \caption{Time domain vs. frequency domain.}
    \label{time_vs_frequency_comparison}
\end{figure}

In the digital images world, the frequency domain is used to know by how much pixels variate from one another, in other words, the rate with which the pixels change. The most common format used for images that takes advantage of the frequency domain is JPEG, but since it is the only known format which uses sinusoids when rendering the image, the authors chose not to include it since the steganographic surface was somewhat limited. However, in the digital audio world every sound will eventually become an analog signal before reaching our ears so it is more common to see algorithms developed specifically for this format that involve altering the sinusoids to our advantage.

For example, we have the technique described in the article Frequency Domain Steganography by Ganier et al.\cite{ganier_hollman_rosser_swanson} where they showcase the most basic way of embedding an audio file within another audio file: since both files have audio signals that are stored as frequencies, it is possible to "compress" the signals so that they only occupy a very specific frequency range and hide the message within the inaudible frequencies of the carrier, a much trivial task when not working in the time domain. This method takes advantage of the human physiology we mentioned earlier and achieves a high rate of success. Similar work has been done by Westfeld et al. \cite{dsss_sstv} where they took the audio signal generated by the Slow Scan Television(SSTV) and embedded it into another carrier audio signal without any audible noise being generated that could alert intruders. On the more technical and software side of things there are applications such as Audacity that can integrate plugins written in a language called Nyquist that are specifically designed for frequency encoding secret messages, along with Matlab implementations of the aforementioned papers and many more.

Furthermore, there is also the option of steganography done within the spectrogram of a signal. A spectrogram is the visual represention of the entire spectrum of the frequencies as it evolves over time, basically getting the frequency domain and reintroducing the time axis into it. It is by far the most common place of hiding messages because it has been popularized by Capture the Flag competitions as entry level challengs and easter eggs in the video game community created by the developers. An example of hiding a key inside the spectrogram of an audio file can be seen in figure \ref{battlefield_spectrogram_easter_egg}, made by the developers at DICE for their community in a secret challenge\cite{battlefield wiki}.

\begin{figure}[H]
    \centering
    \includegraphics[width=7.7cm,keepaspectratio]{pics/audio_chapter/spectrogram_encoding.jpg}
    \caption{Message within the spectrogram viewed using Audacity.}
    \label{battlefield_spectrogram_easter_egg}
\end{figure}


\section{Waveform Audio (WAV)}
The Waveform Audio format commonly known as WAV is a popular file format for storing high quality digital audio files originally built by Microsoft. It bears many similarities to the PNG format in the internal structure/composition of the file: both begin with a very specific sequence of bytes (also known as the magic bytes) that help classification applications identify them, are separated into multiple parts (also known as chunks) that have their purpose and are extremely common in the modern day multimedia. WAV is one of the most common file formats that can be found in the digital world and that makes it a viable candidate for the message carrier role in the steganography domain. Like almost all other formats designed by Microsoft it has a few variations and extensions of the original specifier to accomodate more metadata or higher quality audio, but since they are much rarer than the simple and original standard, we will not focus on those altered versions in this thesis.

In the table below we can see the general structure of almost any WAV file that contains PCM data \footnote{PCM is an abbreviation for Pulse-code modulation which is the most common way of storing digitally the sampled analog audio signals.} which are basically the default option of every digital audio recording software. From the table we can notice how there aren't any metadata fields that are less important which could make for a good initial foothold in the steganography process.
\begin{center}
\begin{tabular}{|c|c|c|}
\hline
\textbf{Chunk field} & \textbf{Field Length} & \textbf{Description} \\ \hline
Chunk ID & 4 bytes & \begin{tabular}[c]{@{}c@{}}Always equal\\ to "RIFF"\end{tabular} \\ \hline
Chunk Size & 4 bytes & \begin{tabular}[c]{@{}c@{}}Length in bytes\\ of remaining file\end{tabular} \\ \hline
Format & 4 bytes & \begin{tabular}[c]{@{}c@{}}Always equal\\ to "WAVE"\end{tabular} \\ \hline
Subchunk ID & 4 bytes & \begin{tabular}[c]{@{}c@{}}Always equal \\ to "fmt "\end{tabular} \\ \hline
Subchunk size & 4 bytes & \begin{tabular}[c]{@{}c@{}}Always equal\\ to 16\end{tabular} \\ \hline
Audio Format & 2 bytes & Equal to 0x0001 \\ \hline
Nr. of channels & 2 bytes & \begin{tabular}[c]{@{}c@{}}Channel count\\ (1 mono, 2 stereo etc.)\end{tabular} \\ \hline
Sample Rate & 4 bytes & \begin{tabular}[c]{@{}c@{}}How many samples\\ per second(6000Hz,\\ 44100Hz, etc.)\end{tabular} \\ \hline
Byte Rate & 4 bytes & \begin{tabular}[c]{@{}c@{}}Product of SampleRate,\\ number of channels and\\ the byte per sample\\ (BitsPerSample/8)\end{tabular} \\ \hline
Block Align & 2 bytes & \begin{tabular}[c]{@{}c@{}}Actual byte count \\ for a full sample\\ (both channels in a \\ stereo file for example\\ result in 4 bytes for\\ a full sample)\end{tabular} \\ \hline
Bits Per Sample & 2 bytes & \begin{tabular}[c]{@{}c@{}}How many bits\\ in a sample for\\ a single channel\end{tabular} \\ \hline
Subchunk ID & 4 bytes & Equal to "data" \\ \hline
Subchunk size & 4 bytes & Length of actual data \\ \hline
Data & * & The actual data \\ \hline
\end{tabular}
\end{center}

However, we see a great deal of advantages as well based on the file structure:
\begin{itemize}
	\item \textbf{Uncompressed data} means that we do not have to go through the process of decompressing it, editing it, and then recompressing it in order to store any information. We are able to go sequentially through each byte and without being careless we can alter it in order  to perform least significant bit insertion and hide the message, making WAV a good carrier for any type of sequential steganography. There are a few issues that could arise, usually samples are stored on two bytes and altering the least significant bit of each byte will alter the associated sample in a much much greater way, so we must keep in mind the bit depth of the samples when inserting the data. Furthermore, we must be careful with the audio channels of the file since modifying only one channel will possibly introduce noise to the audio stream, raising red flags to intruders and blowing the cover of the carrier.
	\item \textbf{The chunked structure} of the format leads to the ability to insert custom chunks with our data because most parsers will simply ignore them and look for the relevant chunks that they need. There are also a few types of chunks such as INFO or JUNK that could be used for a safer and more covert communication because they are a part of the standard and that means there will be no risk of crashing the parser if they find a chunk they could not identify.
	\item \textbf{Length is known beforehand} means that we know where the audio data ends just by parsing a few relevant metadata fields. This means that it is very easy to compute the offset where the file ends and we can begin embedding secrets while avoiding breaking the parsers or disturbing the audio data.
\end{itemize}

In conclusion, WAV is an impressive carrier format worthy of being used in the steganography process due to the fact that it is commonly seen on the Internet, has no compression that can affect the secret message and can easily work with all the algorithms presented over the course of this thesis.

\section{The MPEG-1/2 Audio Layer III (MP3)}
Moving Pictures Experts Group or MPEG for short has plenty of both official and unofficial standards, multiple versions etc. The focus of this subchapter will be on MPEG Versions 1 and 2 (the only officially accepted standards), or to be more precise, Layer III of these versions. As you can already see, there are tons of variations of what should  be a single and standardized format, but all of them exist for a good reason and have their own use cases, even though they differ in available bit rates or the sampling rate frequencies range. Talking about all of them would be pointless and would easily take hundreds of pages to showcase as they have been the result of many years of research and evolution, so instead we are going to focus on the most common standard that is available in the digital world, which is even more popular than the aforementioned audio file format, and that is MPEG Version 1 Layer 3, hereby known simply as MP3. While technically any version of the MPEG that uses Layer 3 is formally known as MP3, versions 2 and 2.5 are much rarer since they offer smaller bitrates and smaller frequency sampling rates, making them inadequate for audio usage in the modern day.

The biggest reason MP3 has been one of the most popular audio file formats used in the world since its invention back in 1993 is due to its ability to replay high quality audio samples while still keeping a very small disk usage overhead. It takes advantage of the Huffman coding to reduce the total length of the file, making it able to retain the same perceptible audio quality to the human ear while still compressing the final size between 75\% and 95\% of other uncompressed formats, such as WAV\cite{genesis_of_mp3}. However, due to the usage of compression algorithms, the long time the format was in development, the high number of standards and substandards and the inclusion of other formats in the same binary stream as the audio samples for various reasons, make the MP3 an incredibly complex format that need more advanced parsers than the ones found in other trivial formats, such as BMP or WAV.

Before beginning to talk about steganographic algorithms that could be applied to the format, we first need to understand the MP3 binary stream to make sure we do not alter bytes that could mark the file as corrupted, such as checksums or important headers. Nowadays, the structure of most MP3 files consists of two parts:
\begin{itemize}
	\item The \textbf{ID3} part, the metadata container where information such as the track artist, album name or even album art is stored.
	\item The actual \textbf{MP3} audio data, contains the audio samples in a special format.
\end{itemize}

ID3 is a standard used to store file metadata that is most commonly seen alongside MP3 files, which is why it is important to discuss it because it can offer a few interesting options that can be viable secret message carriers. It is always found at the beginning of the audio file such that any parsers that are not interested in the metadata will need to read only a few bytes in order to get the offset to the actual sound data. Similar to the PNG, it is a chunked structure that may contain multiple chunks (or frames as the standard specifies). We can see in table below the general structure of the ID3v2 header(the most common type of header, the previous versions being flagged as obsolete):

\begin{center}
\begin{tabular}{|c|c|}
\hline
\textbf{Chunk type}                                           & \textbf{Chunk size}                                       \\ \hline
\begin{tabular}[c]{@{}c@{}}Header\\ (REQUIRED)\end{tabular}   & 10 bytes                                                  \\ \hline
\begin{tabular}[c]{@{}c@{}}Extended Header\\ (OPTIONAL)\end{tabular} & \begin{tabular}[c]{@{}c@{}}Variable\\ Length\end{tabular} \\ \hline
\begin{tabular}[c]{@{}c@{}}Frames\\ (at least 1)\end{tabular} & \begin{tabular}[c]{@{}c@{}}Variable\\ Length\end{tabular} \\ \hline
\begin{tabular}[c]{@{}c@{}}Padding\\ (OPTIONAL)\end{tabular}  & \begin{tabular}[c]{@{}c@{}}Variable\\ Length\end{tabular} \\ \hline
\begin{tabular}[c]{@{}c@{}}FOOTER\\ (OPTIONAL)\end{tabular}   & 10 bytes                                                  \\ \hline
\end{tabular}
\end{center}

We can see from the table the similarities to other chunked file formats. Since modifying any of the important bytes from the header/footer or padding can lead to corrupted and unparsable files, we shall stray away from the. However the frames are a perfect environment for our messages since we can have almost any number of them (limited by the 4 bytes or 28 bits that form a synchsafe integer representing the total size of the ID3 header), along with a high range of official frame types that we can pick from that most parsers will probably ignore. We can see in the figure below a frame with the ID equal to APIC (short for attached picture) which is used to usually embed images into the stream that represent album art or covers.
\begin{figure}[H]
    \centering
    \includegraphics[height=8cm,keepaspectratio]{pics/audio_chapter/tool_cover_example.png}
    \caption{Front cover of the album Fear Inoculum by Tool}
\end{figure}

Our advantage is that there is no limit of frames with the ID equal to APIC there can be in a file so it is a prime target for metadata steganography when our message is a simple image. By changing a few bits we can mark the image as being the back cover of the song or we can even make it invalid such that it will exist in the binary stream but no renderers will show it. The process of parsing the metadata is simple after reading and understanding the documentation and results in a trivial encoding process. The difficult part is when attempting to write our image since it has to be inserted without breaking any of the other frames. After the entire process is done, we get something like this:
\begin{figure}[H]
    \centering
    \includegraphics[height=8cm,keepaspectratio]{pics/audio_chapter/tool_backcover_example.png}
    \caption{Back cover which is our hidden image}
\end{figure}

Unfortunately, since MP3 is a much more complex format than we are accomodated with we will not look into the process of Least Significant Bit insertion as that would entitle too much trouble for such a simple result and since it is by default a compressed format it is not a significant loss because the available domain is smaller and we would not be able to achieve an acceptable performance. This is the reason why metadata and after-end steganography are the only viable methods when dealing with the MP3 file format.

In conclusion, MP3 is one of the best formats for metadata embedding since it has so many types of frames that are impossible to keep track of and that no user will possibly notice if they are not looking specifically for it. 
\end{multicols*}

\chapter{The Steganos Project}
\begin{multicols}{2}
Steganos is the name of the application that implements the algorithms enumerated in the earlier chapters of this paper and is the direct result of the work done by the authors. In this chapter I will present what modern technologies and frameworks went into creating the application, how it is structured from an Object Oriented point of view, how fast it is and how I hope it will expand into the future.

\section{Used technologies}
\subsection{C++}
C++ is one of the lowest high-level computer programming languages. Designed by Bjarne Stroustrup, it first appeared in 1985 as a variation of one of the most popular languages at the time, C. It is one of the most efficient modern languages mainly because it has been designed with performance and flexibility in mind, just like its predecessor. C++ has gained a lot of traction  from big companies like Intel and Microsoft which needed a programming language that could be used for operations ranging from basic kernel functions to highly specialized Object-Oriented projects with Graphical User Interfaces\cite{stroustrup_2018}.

Since its conception, C++ has grew substantially by adding support for generic and functional features to ease the development processes and getting standardized by the International Organization for Standardization probably helped as well because it meant no more obscure variations of the language, allowing programmers to follow only one standard, ending up with even more portability and stability of the applications.

In the modern day, it is used almost everywhere: the kernel of various operating systems, the transaction software used by banks, drones and airplanes, embedded systems such as the Arduino or Raspberry Pi, and now in the Steganos Project as well under the latest approved standard of the language, C++17.

C++ inherited from C the file types: headers and sources. The header files (the files that have a .h or .hpp extension) are where the standard recommends to store the class declarations, function prototypes, constants declarations, no definitions should take place in a header file. There are a few exceptions to this rule, the most common one being a rule that states that all templated functions and functions are to be declared and defined in a header file otherwise it will not work properly when dealing with multiple instances of different types. In the case of the source files (the files that have the .c or .cpp extension) the standard suggests to only be used for the definition of the class methods or other general functions. Following the standard results most of the time in a clean and well defined separation in the code, allowing for high cohesion and low coupling in all projects built using C++.

The reason for choosing C++ as the language for the Steganos project is simple: it allows for the programmer to work on the raw bytes of files in an extremely easy way, reading and parsing them, bit-wise operations and writing to file, all the common low-level operations that are highly valued when working in the steganography field. But the high level aspects of the language also permit for dealing with complex tasks, such as modern cryptography algorithms applied to robust byte streams or holding all the information in different data structures.

\subsection{CMake}
CMake is a cross-platform open-source tool designed for managing the build process of software based on the C++ programming language. It has been created in the year 2000 and ever since then it stayed compiler-independent using simple configuration files that generate the adequate makefiles to be used in the users environment while building the target project\cite{cmake}.

CMake uses files called CMakeLists.txt that contain the commands to be used by the internals of the tool in the building process. It is required that one file is in the root of the project before running the CMake process, with the possibility of adding a .txt file in the subdirectories in order to indicate special cases that require a different approach.

\begin{figure}[H]
    \centering
    \includegraphics[height=6.9cm,keepaspectratio]{pics/cmake_folder_structure_example}
    \caption{Folder structure of a project using CMake}
\end{figure}

Each command in CMake has the same format: COMMAND (args..). Using this format, users are able to build even the most complex software projects in the form of simple executables, dynamic or static linked libraries, etc. Steganos uses CMake because it is an extremely effective tool in the building process and it allows for separating the logic of the project into distinct modules i.e. the audio module, the image module, the general usage module, and linking them in the end into a simple executable. Most projects use only a small subset of the commands available in CMake, the most common being:

\begin{itemize}
  \item \textbf{CMAKE\_MINIMUM\_REQUIRED} is used to mark the minimum version of CMake that is required to be installed on a system in order to build the project.
  \item \textbf{SET} is used to assign a value to a CMake variable that is used while building, similar to environment variables found on all operating systems.
  \item \textbf{PROJECT} for naming the project, important step in active Continous Integration and Continous Development environments.
  \item \textbf{INCLUDE\_DIRECTORIES} for marking the directories containing the header files in order for the compiler to be able to link the source files and header files.
  \item \textbf{ADD\_EXECUTABLE} for creating an executable after build the project from the source files given as arguments.
  \item \textbf{ADD\_LIBRARY} for creating a library or module with the given source files.
  \item \textbf{TARGET\_LINK\_LIBRARIES} for linking (pre)defined modules or executables between eachother.
\end{itemize}


\subsection{CXXOpts}
CXXOpts is an open-source C++ library that is meant to be a lightweight command line option parser\cite{jarro2783_2020}. It is meant as the C++11 and beyond alternative to other libraries such as Commons CLI for Java or Argparse for Python. It was initially created by user jarro2783 on Github and nowadays is actively developed by the community using the Git commit and pull request system. CXXOpts is made to replicate the handling of the help argument and the merging of multiple parameters commonly found in *NIX command line binaries, without using any additional dependencies in a header-only file.


\subsection{Lodepng}
Lodepng is an open source C++ library developed by Lode Vandevenne used for image processing for pictures stored under the PNG format \cite{lvandeve_2020}. It is very useful for decoding the image data before beginning the steganography process, making it easy to modify the data and to encode it back into a PNG file that follows the standard format. Lodepng works on every version of C++, contains only two files(a source file and a header file) and is rated to be one of the fastest libraries for PNG image processing. Steganos uses Lodepng to parse PNG files and to obtain the image data byte stream from the zlib compressed stream in order to be able to work with the implemented steganographic algorithms.

\subsection{Robot36 SSTV Engine}
Robot36 is a library created by Ahmet Inan in order to be able to encode and decode images using the Slow Scan Television (also known as SSTV) format, more specifically the Robot36 standard. The library is open-source and is available on github\cite{robot36_git}. By default it supports only color BMP images that are 320 pixels wide and 240 pixels tall and can be compiled using any C compilers only under a Linux system which has the Advanced Linux Sound Architecture (ALSA) libraries installed. Since Steganos is a C++ project that the authors intended to be cross-platform, we have forked the original project and refactorized the code to be buildable and to work on the Windows architecture as well. After some more refactorization is done, we have every intention to release our version of the SSTV Engine and to create a pull request to the original repository.

\begin{figure}[H]
    \centering
    \includegraphics[height=7.5cm,keepaspectratio]{pics/application_chapter/sstv_example_iss}
    \caption{SSTV Engine decoding transmission from the International Space Station}
\end{figure}

\subsection{Jenkins}
Jenkins is one of the biggest open source automation servers that is meant to automate the process of building, deploying, testing etc. any project. Developed in Java, it supports a big range of popular languages and frameworks, either by default configuration or by installing a plugin available on their developer page. We picked Jenkins for Steganos due to previous working experience of the authors with it and since it has the ability to build and release C++ and CMake projects using a single plugin it has proven to be an extremely valuable asset in maintaning a continuous integration and continuous delivery schedule of the project.

\begin{figure}[H]
    \centering
    \includegraphics[height=7.5cm,keepaspectratio]{pics/application_chapter/steganos_jenkins_job}
    \caption{Jenkins management interface of the Steganos job}
\end{figure}


\section{Application architecture}
The Steganos Project is an application that uses the Object Oriented Programming aspects in order to achieve the best performance while maintaining high cohesion and low coupling between the components of the codebase. The project is structured in multiple modules\footnote{To be technically correct they are only different folders because the latest available C++ standard as of the time of writing this thesis is C++17 which does not understand yet the concept of modules. However developers should rejoice over the announcement made by Microsoft to standardize C++ modules starting with the C++20 standard\cite{N4720}.} that were built to be as independent as possible with the goal to provide steganography functions for the supported file formats. The main module contains the source files for the entry point of the application which processes the arguments given via the command line along with the code for some of the utilities, helper function that help convert pixels between different representations(YUV, RGB, BGR, YCbCr) or bit-wise operations such as toggling the last significant bit or building a byte using the LSBs from a byte stream. The algorithm module declares and defines the algorithms that are implemented in the application, all of which have been discussed thoroughly in the earlier chapters of the thesis (LSB insertion, embedding secrets into file metadata, use-after-end encoding). The remaining modules all follow the same design patterns and are responsible for the parsing of the cover files and the encoding or decoding of hidden digital files inside them. 

We can see in figure \ref{module_overview} the overview of the entire application and what each module contains more specifically. This is only a very simplistic rendering of the module deployment, we will delve in for a more in-depth perspective later this chapter.
\end{multicols}

\begin{figure}[H]
    \centering
    \includegraphics[width=13cm,keepaspectratio]{pics/application_chapter/module_overview}
    \caption{Overview of the Steganos modules}
    \label{module_overview}
\end{figure}

\begin{multicols}{2}
In figure \ref{cover_module_overview} we can see a class diagram of the general struture of each of the four aforementioned cover submodules (the codebase which deals with parsing the cover files that allow for the steganography algorithms to have the necessary data to work with). 
\end{multicols}
\begin{figure}[H]
    \centering
    \includegraphics[width=16cm,keepaspectratio]{pics/application_chapter/diagrama_module_interface}
    \caption{Structure of a cover module}
    \label{cover_module_overview}
\end{figure}

\begin{multicols}{2}
After seeing how the project is structured into modules and submodules, what classes, interfaces, methods and attributes we have, it is also very important to understand how the data flows from the moment we launch Steganos to the moment the output file is generated (the output file is either the cover file containing a secret message, either the actual secret message, depending on the option used). In figure \ref{flowchart-code} we can see a flowchart of the application containing the key points of the entire process.
\end{multicols}
\begin{figure}[H]
    \centering
    \includegraphics[width=15cm,keepaspectratio]{pics/application_chapter/flowchart}
    \caption{Steganos code flowchart}
    \label{flowchart-code}
\end{figure}

\begin{multicols}{2}
To make sure that the project will always be in at least a buildable state after any commit is done we are using Jenkins. It is very important to make sure that the code compiles and that if there are any errors, the authors are aware of them and will have the option to hold back the release until the issues are fixed. On a more related note to the Steganos application, we have the following job configuration for testing each new version before releasing:
\begin{itemize}
	\item Maximum number of build to keep is set to 5 to conserve total storage used.
	\item Each new commit on the master branch will activate the hook trigger to let Jenkins know that a new version is available and that it should try to build it.
	\item Using the "Unix Makefiles" generator to set up the CMake build files with the build type set to "Release".
	\item Publish the build result to the project page to keep it up to date to the current release or trigger custom personal alerts to notify in case of failure.
\end{itemize}
\end{multicols}

\begin{figure}[H]
    \centering
    \includegraphics[width=15cm,keepaspectratio]{pics/application_chapter/jenkins_project_build}
    \caption{Jenkins log of the Steganos job build}
    \label{jenkins-log}
\end{figure}


\definecolor{dkgreen}{rgb}{0,0.6,0}
\definecolor{gray}{rgb}{0.5,0.5,0.5}
\definecolor{mauve}{rgb}{0.58,0,0.82}
\lstset{frame=tb, %configurare listing format pt subcapitolul asta
  language=C++,
  aboveskip=3mm,
  belowskip=3mm,
  showstringspaces=false,
  columns=flexible,
  basicstyle={\small\ttfamily},
  numbers=none,
  numberstyle=\tiny\color{gray},
  keywordstyle=\color{blue},
  commentstyle=\color{dkgreen},
  stringstyle=\color{mauve},
  breaklines=true,
  breakatwhitespace=true,
  tabsize=3
}

\begin{multicols}{2}
\section{Implementation details}
Steganos contains many critical parts that are required to succeed in order to work properly. If any part of code or function call were to fail we would need to know so that we can properly interrupt the process on our terms, free any dynamically allocated memory, log the error messages so that developers are aware of any bugs that may appear or if the application is in a release form, notify the client of what has happened and why the application failed. Since C/C++ is a rather monstrous amalgamation when talking about exceptions raised during runtime execution (at least in the standard libraries, more complex frameworks like Boost do not suffer as much from this), we have preferred to validate the function calls based on a more simple method, the code returned by the function which is basically a glorified integer that has some meaning attached to its values. You can see in listing \ref{lst:try_macro} the exact define macro used in the project and in listing \ref{lst:try_macro_example} an example from one of the encoder module constructors.

\end{multicols}
\begin{lstlisting}[language=C++, caption=The TRY macro used for any critical operation,label={lst:try_macro}]
 #define TRY(X) generic_assert((X), __FILE__, __LINE__);
 inline void generic_assert(error_code code, const char* file, int line) {
   if (code != error_code::NONE) {
     printf("[Assert error]Message = %d in file %s at line %d\n", code, file, line);
     exit(static_cast<int>(code));
   }
 }
\end{lstlisting}

\begin{lstlisting}[language=C++, caption=Usage example of the TRY macro,label={lst:try_macro_example}]
WAVEncoderModule::WAVEncoderModule(const char* cover_file_path, const char* secret_file_path) : WAVModule(cover_file_path) 
{
  TRY(utils::load_stream(secret_file_path, secret_stream));
  TRY(utils::read_byte_stream(secret_stream, secret_data, secret_data_size));
}
\end{lstlisting}


\begin{multicols}{2}
\section{Using the application}
Steganos was designed to be a very simple command line tool that is fast, reliable and portable. Using Cxxopts we have tried to implement an intuitive way of passing the arguments to the program, such that they all make sense and that any of the available configurations and command executions are easy to understand. You can see from the program output below that it is very easy to learn the options available to the end user by adding the -h flag and see what is supported and what not.
\end{multicols}
\begin{verbatim}
D:\Projects\Steganos\out\build\x64-Debug>Steganos.exe
You must specify one cover file.
Do Steganos.exe -h for more information.

D:\Projects\Steganos\out\build\x64-Debug>Steganos.exe -h
Simple steganography project created as a part of my bachelor's thesis
Usage:
  Steganos [OPTION...]

 Encoding/Decoding options:
  -c, --cover FILE   The file that serves as a cover for the secret message
  -s, --secret FILE  The secret file which will be embedded into the cover
                     file - ENCODING ONLY

 General options:
  -o, --output FILE             Name of the output file (default: output)
  -v, --verbose                 Verbose output
  -m, --method METHOD           The method to be used when encoding/decoding
                                the message
  -p, --passkey PASSKEY         The password used to secure the message or to
                                decipher it (default: password)
  -h, --help                    Prints this help message
\end{verbatim}

\begin{multicols}{2}
Based on the information above, it is almost trivial to come up with working command examples based on the help menu we displayed earlier. We can begin looking at some examples to see how Steganos works. Let's assume that we want to encode a secret message stored in a txt file within a BMP file. All we would need to type is the path to the secret file and the path to the cover file. The output can be seen below.
\end{multicols}
\begin{verbatim}
D:\Projects\Steganos\out\build\x64-Debug>Steganos.exe --cover marbles.bmp --secret secret.txt
Done
\end{verbatim}

\begin{multicols}{2}
That's all there is to it. However to some the need to type so many characters without any autocomplete (because option autocomplete is not commonly available in most shells) could be an annoyance so Steganos also offers the option for shortened option names to make the process quicker.
\end{multicols}
\begin{verbatim}
D:\Projects\Steganos\out\build\x64-Debug>Steganos.exe -c marbles.bmp -s secret.txt
Done
\end{verbatim}

\begin{multicols}{2}
We also have the option to enable verbose output to see debugging information in real time during the Steganos process and see exactly what it is doing. We can also specify the name of the output file to be whatever we would like it to be.
\end{multicols}
\begin{verbatim}
D:\Projects\Steganos\out\build\x64-Debug>Steganos.exe -c marbles.bmp -s secret.txt 
                                      \\ -o super_secret_marbles.bmp -v
BMP Image data is found beginning at 54
Ignoring 0 bytes to get to the actual image data
Available methods:
SEQUENTIAL (default)
PERSONAL_SCRAMBLE
No encoding method specified! Picking default option.
Deleted image data and its associated stream
Done
\end{verbatim}

\begin{multicols}{2}
\section{Further work}
Regarding the future of Steganos, I hope that it will be fairly fruitful and worthwhile. The entirety of this thesis we have only laid down the groundwork of the project and have talked about the most basic and essential part of the application. However, the possibilities for extending Steganos are almost endless and during the time while building the application and writing this thesis I have saved some of the more note-worthy ideas that came to my mind (in no particular order):
\begin{itemize}
	\item \textbf{Additional file formats support}. Right now Steganos only supports 4 major file formats, two digital image formats (BMP and PNG) and two audio formats (WAV and MP3). However, there are a lot more file formats that are commonly used on the Internet and that could be viable picks as a cover in the steganography process. Some of those formats include but are not limited to: Joint Photographic Experts Group (JPEG images) format, Free Lossless Audio Codec files (high-quality audio files, commonly found on album disks), Graphic Interchange Format (GIFs) and many more. 
	\item \textbf{Proxy server and chat application}. This is one of best development Steganos could get in my opinion. As mentioned earlier, right now the project is similar to only a core-engine library that just takes some input, processes it and turns it into an output. It is not integrated anywhere, it's just a simple CLI application. However, it would be nice to see it integrated into a chat application, following a very simple idea: run Steganos as a server process which accepts any kind of message and returns a multimedia file that has the secret embedded within and then send that image/audio/video file to the peers. They would be able to decode the message based on an already negociated process (the negociation took over a secure channel that any intruders would not have access to) and they would be able to reply in the same manner. The clients could also change the encoding/decoding process automatically and then notify the peers by also embedding the renegociation in the cover file. This application has the advantage that it only needs a brief secure environment for the initial connection and then be able to securely coomunicate over any type of channel. It would certainly be one of the most interesting usages for a steganography project, mostly because it would be practical enough to be used in real world situations and not just Capture the Flag challenges.
	\item \textbf{Windows integration for Jenkins}. At the moment of writing this, there is only one deployed instance of Jenkins running the Steganos job and it is based on Linux. That means that the only types of files that are automatically generated and tested are only the Linux binaries since there is no easy way of also building the Windows binaries. Some research has revealed that there is the option of adding a Windows based Jenkins instance in a master-slave configuration that would technically allow the full building and testing of the executables but we have not yet managed to implement this. 
	\item \textbf{Additional algorithms}. Most algorithms described in this paper are implemented in the Steganos project but there are plenty viable ones (especially those that rely on scrambling the data) that have not seen yet the light of day. It would certainly be interesting to add them and compare them to existent implementations or to the other methods.
	\item \textbf{Encrypting the secret}. Even though the option is specified in the help menu, it is not yet possible to encrypt the actual secret, mainly because I could not decide on which encryption methods to make available and because I also would rather focus on building the core codebase and to implement the steganography algorithms first. However, it is almost mandatory that the option of encrypting the secret before the embedding process is added in a future release to increase the security of the contents.
\end{itemize}
\end{multicols}

\clearpage
\chapter{Conclusion}
In conclusion, I believe that steganography in modern multimedia is in a really interesting position: it is not advanced enough to win over cryptography and become the main pillar behind secure communications over unreliable networks and most likely this will never change since steganography by definition will always be slower because it relies on a carrier for securing the message instead of just securing the message itself. However is steganography still an extremely important method of protecting information such that only the entrusted parties that are aware of the used algorithms will be able to extract the hidden message. It is my belief that there are countless more steganographic innovations to be made in this field that researchers will discover in the future, expanding the number of applications that steganography has right now beyond digital watermarking or Capture The Flag challenges in hacking competitions. 

Taking a step back to look at the entire known history of steganography methods we can clearly see that it has evolved a lot: from the Roman Empire hiding messages on the shaved heads of their slaves, waiting for their hair to grow back and then send them away with important messages, to the modern day digital steganography that contains several impressive and complex algorithms in order to ensure high capacity and fast encoding/decoding processes. Given this evolution it is same to assume that more and more complex and well researched algorithms will appear in the future, bringing steganography closer to the spotlight of various information security topics and conferences.

\newgeometry{top=-0.3in}
\bibliographystyle{acm}
\bibliography{chapters/references}
\addcontentsline{toc}{chapter}{Bibliography}
\restoregeometry
\end{document}